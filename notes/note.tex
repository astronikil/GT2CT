\documentclass{article}
\usepackage{graphicx} % Required for inserting images
\usepackage{amsmath,amsfonts,amssymb,slashed}
\usepackage{physics}
\usepackage[backend=biber,style=numeric-comp,sorting=none]{biblatex}
\addbibresource{Refs_Bahrgava.bib}
\usepackage[letterpaper,top=2cm,bottom=2cm,left=3cm,right=3cm,marginparwidth=1.75cm]{geometry}
\newcommand\beq{\begin{equation}}
\newcommand\eeq[1]{\label{#1}\end{equation}}
\newcommand\beqa{\begin{eqnarray}}
\newcommand\eeqa[1]{\label{#1}\end{eqnarray}}
\newcommand\eqn[1]{Eq.\ (\ref{#1})}

\usepackage{hyperref}
\hypersetup{
   colorlinks=true,
   linkcolor=blue,
   filecolor=magenta,
   urlcolor=cyan,
}

\title{Genetic Tool to Cell Type Bayesian Mapper (GT2CT Bayesian Mapper)}
\author{Nikhil Karthik and Chaitali Khan}
\begin{document}

\maketitle

Genetic Tool to Cell Type Bayesian Mapper (GT2CT Bayesian Mapper) is a python based toolkit for 
finding the likely fractions of brain cell-types (class, subclass, supercluster or cluster)
present in the GFP/RFP positive cells tagged by genetic tools.

In the note below, we will use "VISp" to specify a region of brain, but it is just a placeholder for any 
choice of brain region parcellation.
We will use the term "subclass" to specify cell-type, but it is a placeholder for
cell-type at class, subclass, supercluster or cluster. The tool kit is written in a generalized way 
to perform analysis in any brain region and cell-type classification levels. But, we 
have only tested the algorithm in the VISp at subclass level at the time of writing.

\section{Mathematical background}

\subsection{Notation}

Let $\mathbf{x}=(x, y, z)$ be the three-dimensional CCF coordinates in the VISp area.  The brain slices 
are labeled by constant $z$ values.  Let $S$ be the set of subclasses present in VISp area, and let 
$N$ be the size of this set of subclasses. That is, for VISp, $S =$ \{L6 CT CTX Glut, Pvalb Gaba,$\ldots$\} 
with 28 other subclasses in the set. Note that this information on cell-type is inferred from the MERFISH cell data.

Next, we have the GFP/RFP positive cells tagged by a gene tool. 
Let $K$ be the target specificity of the gene tool. Either one knows
how many $K$ cell types can be tagged by the genetic tool, or one
knows a prior distribution over $K$ for gene tools used. For now,
we assume that the value of $K$ is known, and later we will use
prior knowledge that $K$ is small. Given the value of $K$, there
are $\binom{N}{K}$ subclass combinations possible.
Let ${\cal K}$ is the set of these $\binom{N}{K}$ subclasses $K$-plets, and
let its members be $k$. These $K$-plets $k$ are, for example,
such as $k_1 =$ (Oligo, Astro, Sst Gaba), $k_2=$(L5 ET CTX Glut, Sncg Gaba, Microglia NN), etc., for 
$K=3$.

\subsection{Probability distributions}

Let $P(s|\mathbf{x})$ be the probability to find a cell of type $s\in S$
at position $\mathbf{x}$. 
The net probability to find any of the subclasses at a location $\mathbf{x}$ should be one; that is
\beq
\sum_{s\in S} P(s|\mathbf{x}) = 1.
\eeq{norms}
Let $P'(k|x)$ be the
probability to find a subclass $K$-plet $k \in {\cal K}$ at CCF location $x$.  This is
related to the distribution of individual subclass $P(s|\mathbf{x})$ as
\beq
P'(k|\mathbf{x}) \propto \sum_{s\in k} P(s|\mathbf{x}),
\eeq{kdist}
with the proportionality constant that correctly normalizes $P'(k|\mathbf{x})$. We also want
the spatial distribution of a $K$-plet $k$ given by $q(\mathbf{x}|k)$. This is given by the Bayes theorem to be 
\beq
q(\mathbf{x}|k) = \frac{P'(k|\mathbf{x})}{\sum_{\mathbf{x}'} P'(k|\mathbf{x}') },
\eeq{bayes}
assuming a uniform prior distribution over position. For example, let us say that we know that a 
genetic tool specifically targets $k=$(L5 ET CTX Glut, Sncg Gaba, Microglia NN).  Then, the 
distribution $q(\mathbf{x}|k)$ gives the spatial distribution to find one of the members of $k$ 
to be at various $x$.

Let $Q_{\rm GFP}(\mathbf{x})$ be the spatial distribution of GFP/RFP cells. If one had a really high resolution
image of GFP/RFP cells, and one knew with 100\% confidence which were GFP/RFP cells versus the background, 
then we can still describe the cells through a distribution $Q_{\rm GFP}(\mathbf{x})$ that are non-zero only 
at exact locations of the GFP cells. In general, this is not the case, and we take the approach to assign 
only a continuous spatial distribution $Q_{\rm GFP}(\mathbf{x})$ that there is a GFP/RFP cell at various $\mathbf{x}$.
We will define our procedure to assign $Q_{\rm GFP}(\mathbf{x})$ based on intensity distribution later in this note.

\subsection{Mathematical statement of the algorithm}
The basis of our algorithm is the following.  Given $Q_{\rm GFP}(\mathbf{x})$ of the GFP/RFP cells via the STPT images 
of the brain, given a spatially smooth distribution $P(s|x)$ of the celltypes inferred from MERFISH data, and 
given that the gene tool can tag only certain cell types from $K$-plets $k$ all over VISp, we find the single best $K$-plet $k$ 
whose $q(\mathbf{x}|k)$ is the most similar to $Q_{\rm GFP}(\mathbf{x})$.

Such a measure of similarity between two probability distributions $p$ and $q$ is given by the cross-entropy ${\cal H}(p,q)$ which is a minimum only if $p=q$. For our case, we want to find $k$ such that 
\beq
{\cal H}(k) = - \sum_{\mathbf{x}\in\text{VISp}} Q_{\rm GFP}(\mathbf{x}) \log\left[ q(\mathbf{x}|k)\right],
\eeq{cross}
is minimized among all $k\in {\cal K}$. Let $k=k^*$ be the optimal $K$-plet. Then, given the assumption/model that the gene tool has specificity of $K$ cell types, the expected fractions $f_s$ of subclasses $s\in k^*$ present all over VISp is
\beq
f_s =  \sum_{\mathbf{x}\in\text{VISp}} Q_{\rm GFP}(\mathbf{x}) \frac{P(s|\mathbf{x})}{\sum_{s'\in k^*} P(s'|\mathbf{x}) },\quad\text{for}\quad s\in k^*.
\eeq{frac}
For $s\notin k^*$, $f_s=0$.
The assumptions of specificity $K$ and the optimal $k^*$ from ${\cal K}$ are implicit, and it is to be understood that 
$f_s$ is the above equation is actually $f_s\left(k^*(K)\right)$.

\subsection{Model averaging}
The above step concludes the algorithm. But, one could go further to alleviate the model dependence on the choice of gene tool specificity $K$ by a weighted average of $f_s(k^*)$ for different $K$. Studies are needed on how best to perform the weighted average, but for this challenge, we do a simple model averaging as 
\beq
f_s = \frac{1}{K_{\max}} \sum_{K=1}^{K_{\rm max}} f_s\left(k^*(K)\right),
\eeq{modav}
for $K_{\max}=4$.


\section{Implementation of Algorithm}

\subsection{Estimating $P(s|\mathbf{x})$ using multilayer perceptron}

\subsubsection{Details of MLP}

The distribution $P(s|\mathbf{x})$ is a multi-celltype position-based
classification problem.  We wanted to design a protocol which
estimated $P(s|\mathbf{x})$ that can be used in downstream analysis.
We modeled $P(s|\mathbf{x})$ using a simple multilayer perceptron
(MLP) with the input feature being a scaled version $(\tilde{x},
\tilde{y}, \tilde{z})$ of three coordinate components $(x,y,z)$ --
more on scaling function below. The output of the MLP is a 28-way
log-probabilities $\log(P(s|\mathbf{x}))$ for the 28 VISp subclasses.
Between the input and out layers, we used four hidden layers with
64 hidden units each with ReLU activation functions. This simple
setup was found sufficient, and also essential given that the tool
should be usable on a CPU.

\subsubsection{Increasing training data with VISp neighborhood}

For training the weights $w$ of the MLP, we used all the MERFISH
data (coordinates and their subclasses) in the VISp area.  We brought
the MERFISH cells on left section into the right using reflection
symmetry of CCF coordinates. This brought the net trainable data
in VISp to $\sim 62000$.  Next, we increased the trainable data
using cells in VISp's neighborhood. For this, we found the centroid
$(x_{\rm centroid}, y_{\rm centroid}, z_{\rm centroid})$ of the
VISp volume, and quantified the net lateral and longitudinal extend
of the VISp area as
\beqa
\sigma_{xy} &=& \langle \sqrt{(x-x_{\rm centroid})^2 + (y-y_{\rm centroid})^2}\rangle_{\rm VISp},\cr
\sigma_{z} &=& \langle \sqrt{(z-x_{\rm centroid})^2}\rangle_{\rm VISp},
\eeqa{vispsize}
with $\langle \ldots\rangle_{\rm VISp}$ is an average over VISp
space. Then, we included MERFISH cells in the neighborhood of VISp
that were within a radius of $1.5 \sigma_{xy}$ from the VISp's
centroid $z$-axis. This increased the trainable MERFISH cell data
to 211052. This way, we allowed the data slightly outside of VISp
to constrain the $P(s|x)$ in VISp. In order to regularize the numbers
being fed to the MLP, the we input the scaled CCF coordinates
\beq
\tilde{x} = \frac{x-x_{\rm centroid}}{\sigma_{xy}}; \tilde{x} = \frac{y-y_{\rm centroid}}{\sigma_{xy}}; \tilde{x} = \frac{z-z_{\rm centroid}}{\sigma_{z}}.
\eeq{scalepos}

\subsubsection{Training details }

We trained the MLP over 20 epochs with 90\%-10\% train-test split
with a batch size of 128. For the output of MLP to be interpretable
as log-probabilities, we used cross-entropy loss. The model got
trained within 5-6 cycles, and the cross-entropy loss plateaued
around 1.68 on test MERFISH data, and gave a top-3 classification
success rate of $76\%$.

\subsection{Converting STPT data into GFP/RFP probability distributions} 

\subsubsection{Choosing ideal CCF slices in STPT data}

We worked with the STPT data registered to 25 micron CCF.  First,
we chose the optimal $z$-slices of the 25 micron CCF that are closest
to the MERFISH cells that are only available in about every 200
microns.  We wanted to reduce the effect of subdued constraints on
the neural network in CCF regions where MERFISH data are poor or
absent. We included only STPT CCF slices where the average distance
to the neighboring MERFISH cell is less than 0.03 micron within
1-standard deviation.

\subsubsection{Thresholding STPT image}

Next, we removed the diffuse background noise from the STPT data
in the chosen color channel.  We first set $z$-slice dependent
intensity thresholds $I_{\rm thresh}(z)$ such that they encompass
98-percentile of the voxels that contained a nonzero signal in a
$z$-slice. From an examination of OME Zarr image, this choice
approximately reflected the manually chosen thresholds so that one
could just see the individual GFP/RFP cells in many of the cases.
To avoid bias and to avoid sampling from $z$-slices that were dimmer
compared to other areas of VISp, we placed a uniform global threshold
$\bar I_{\rm thresh} = {\rm Median}( I_{\rm thresh}(z) )$. 
Furthermore,
to accommodate left-right asymmetry in some STPT samples, we allowed 
left-right dependence of this global $\bar I_{\rm thresh}$. 
We subtracted this left-right specific global thresholds from all the voxels, 
and set any subtracted intensity less than 0 to be 0.
Since
we did not wish to have few outlier bright GFP/RFP cells to 
be weighted more in our analysis, we placed an upper cutoff of 
$2000$ on the threshold subtracted voxel intensities. 

\subsubsection{Constructing GFP spatial probability distribution}
After the above
thresholding procedure, let these new subtracted intensities be 
$I_{\rm subtracted}(\mathbf{x})$ at various CCF coordinates/voxels $x$. We constructed a 
probability distribution $Q_{\rm GFP}(\mathbf{x})$ to find a GFP cell at $\mathbf{x}$ as 
\beq
Q_{\rm GFP}(\mathbf{x}) \equiv \frac{I_{\rm subtracted}(\mathbf{x})}{ \sum_{\mathbf{x'}\in \text{VISp}} I_{\rm subtracted}(\mathbf{x'}) }. 
\eeq{qxdef}
We use this estimated GFP spatial distribution in \eqn{cross} and
\eqn{frac}. As a practical matter, we sampled $N_{\rm samp} = 80000$ positions
$\mathbf{x}_{\rm samp}$ using distribution $Q_{\rm GFP}(\mathbf{x})$ to evaluate 
weighted sums such as \eqn{cross} and \eqn{frac} as a simple Monte Carlo sampled 
averages.

\subsection{Computing fractions of cell types}

\subsection{Choice of values of typical gene tool target specificity $K$} 

In the last step, we are fitting the best set of $K$ subclasses
that we assume the gene tool targets, and hence are likely to explain
the GFP distribution over all location in VISp. What values of $K$
should one use?  Here, we need to make use of prior information
that gene tool is likely to act specifically on few subclasses.
From Figure-4 of Ben-Simon et al (2025), the value of $K$ are
typically around 3 to 4 cell types with substantial fractions for
many of the gene tools.  Thus we scanned over values of $K$ between
1 to 4. The choice of $K$, in the end, is a trade-off between bias
versus noise.

\subsubsection{Computing cross-entropy between subclass $K$-plet distributions and GFP distributions}

This final step implements \eqn{cross} and \eqn{frac}. The distribution 
$q(x|k)$ determined from \eqn{bayes} using the trained MLP implementation of 
$P(s|x)$. We evaluate \eqn{cross} using sampled GFP coordinates $\mathbf{x}_{\rm samp}$ 
drawn according to $Q_{\rm GFP}$ (denoted $\mathbf{x}_{\rm samp}\sim Q_{\rm GFP}$) as 
\beq
{\cal H}(k) = -\frac{1}{N_{\rm samp}} \sum_{\mathbf{x}_{\rm samp}\sim Q_{\rm GFP} } \log\left[ q(\mathbf{x}_{\rm samp} |k)\right].
\eeq{mccross}
As we noted, there are $\binom{N}{K}$ choices for $K$-plets $k$
that we assume the gene pool can potentially target. The number of
such $K$-plets become large for $K\ge 4$ even for $N=28$, and hence
not feasible to find the value of ${\cal H}(k)$ for all of them.
Therefore, we follow a heuristic procedure to include only the
top-15 subclasses ordered by their cross-entropies (the above
equation with $K=1$) as an effective upper cutoff on $N$, and allow
only $K$-plets formed out of these top-15 subclasses.  We sort the
values of ${\cal H}(k)$, and find the best possible $K$-plet $k^*$
which minimizes ${\cal H}$.

\subsubsection{Computing cell fractions $f_s$}
We find the fractions of celltypes $s$ that are contained with the
best $K$-plot $k^*$ using
\eqn{frac} implemented using the sampled points $\mathbf{x}_{\rm samp}$ as 
\beq
f_s = \frac{1}{N_{\rm samp}} \sum_{\mathbf{x}_{\rm samp} \sim Q_{\rm GFP}} \frac{P(s|\mathbf{x}_{\rm samp})}{\sum_{s'\in k^*} P(s'|\mathbf{x}_{\rm samp}) },\quad\text{for}\quad s\in k^*.
\eeq{mcfrac}
Our algorithm in the Jupyter notebook outputs these fractions in
its penultimate cell for a given choice of $K$.  For this data
challenge, we repeated the last two steps for choices of $K=1,2,3$
and 4, and performed a model averaging of $f_s$ using \eqn{modav}.


\end{document}

