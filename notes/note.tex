\documentclass{article}
\usepackage{graphicx} % Required for inserting images
\usepackage{amsmath,amsfonts,amssymb,slashed}
\usepackage{physics}
\usepackage[backend=biber,style=numeric-comp,sorting=none]{biblatex}
\addbibresource{Refs_Bahrgava.bib}
\usepackage[letterpaper,top=2cm,bottom=2cm,left=3cm,right=3cm,marginparwidth=1.75cm]{geometry}
\newcommand\beq{\begin{equation}}
\newcommand\eeq[1]{\label{#1}\end{equation}}

\usepackage{hyperref}
\hypersetup{
   colorlinks=true,
   linkcolor=blue,
   filecolor=magenta,
   urlcolor=cyan,
}

\title{Genetic Tool to Cell Type Bayesian Mapper (GT2CT Bayesian Mapper)}
\author{Nikhil Karthik and Chaitali Khan}
\begin{document}

\maketitle

Genetic Tool to Cell Type Bayesian Mapper (GT2CT Bayesian Mapper) is a python based toolkit for 
finding the likely fractions of brain cell-types (class, subclass, supercluster or cluster)
present in the GFP/RFP positive cells tagged by genetic tools.

In the note below, we will use "VISp" to specify a brain volume, but it is just a placeholder for any 
brain volume parcellation.  We will use the term "subclass" to specity cell-type, but it is a placeholder for
cell-type at class, subclass, supercluster or cluster. This makes the discussion focus on the challenge.

\section{Mathematical background}

\subsection{Notation}

Let $\mathbf{x}=(x, y, z)$ be the three-dimensional CCF coordinates in the VISp area.  The brain slices 
are labelled by constant $z$ values.  Let $S$ be the set of subclasses present in VISp area, and let 
$N$ be the size of this set of subclasses. That is, for VISp, $S =$ \{L6 CT CTX Glut, Pvalb Gaba,$\ldots$\} 
witb 28 other subclasses in the set. Note that this information on cell-type is infered from the MERFISH cell data.

Next, we have the GFP/RFP positive cells tagged by a gene tool. 
Let $K$ be the target specificity of the gene tool. Either one knows
how many $K$ cell types can be tagged by the genetic tool, or one
knows a prior distribution over $K$ for gene tools used. For now,
we assume that the value of $K$ is known, and later we will use
prior knowledge that $K$ is small. Given the value of $K$, there
are $\binom{N}{K}$ subclass combinations possible.
Let ${\cal K}$ is the set of these $\binom{N}{K}$ subclasse $K$-plets, and
let its members be $k$. These $K$-plets $k$ are, for example,
such as $k_1 =$ (Oligo, Astro, Sst Gaba), $k_2=$(L5 ET CTX Glut, Sncg Gaba, Microglia NN), etc., for 
$K=3$.

\subsection{Probability distributions}

Let $P(s|\mathbf{x})$ be the probability to find a cell of type $s\in S$
at position $\mathbf{x}$. 
The net probability to find any of the subclasses at a location $\mathbf{x}$ should be one; that is
\beq
\sum_{s\in S} P(s|\mathbf{x}) = 1.
\eeq{norms}
Let $P'(k|x)$ be the
probability to find a subclass $K$-plet $k \in {\cal K}$ at CCF location $x$.  This is
related to the distribution of individual subclass $P(s|\mathbf{x})$ as
\beq
P'(k|\mathbf{x}) \propto \sum_{s\in k} P(s|\mathbf{x}),
\eeq{kdist}
with the proportionality constant that correctly normalizes $P'(k|\mathbf{x})$. We also want
the spatial distribution of a $K$-plet $k$ given by $q(\mathbf{x}|k)$. This is given by the Bayes theorem to be 
\beq
q(\mathbf{x}|k) = \frac{P'(k|\mathbf{x})}{\sum_{\mathbf{x}'} P'(k|\mathbf{x}') },
\eeq{bayes}
assuming a uniform prior distribution over position. For example, let us say that we know that a 
genetic tool specifically targets $k=$(L5 ET CTX Glut, Sncg Gaba, Microglia NN).  Then, the 
distribution $q(\mathbf{x}|k)$ gives the spatial distribution to find one of the members of $k$ 
to be at various $x$.

Let $Q_{\rm GFP}(\mathbf{x})$ be the spatial distribution of GFP/RFP cells. If one had a really high resolution
image of GFP/RFP cells, and one knew with 100\% confidence which were GFP/RFP cells versus the background, 
then we can still describe the cells through a distribution $Q_{\rm GFP}(\mathbf{x})$ that are non-zero only 
at exact locations of the GFP cells. In general, this is not the case, and we take the approach to assign 
only a continous spatial distribution $Q_{\rm GFP}(\mathbf{x})$ that there is a GFP/RFP cell at various $\mathbf{x}$.
We will define our procedure to assign $Q_{\rm GFP}(\mathbf{x})$ based on intensity distribution later in this note.

\subsection{Mathematical statement of the algorithm}
The basis of our algorithm is the following.  Given $Q_{\rm GFP}(\mathbf{x})$ of the GFP/RFP cells via the STPT images 
of the brain, given a spatially smooth distribution $P(s|x)$ of the celltypes inferred from MERFISH data, and 
given thet the gene tool can tag only certain cell types from $K$-plets $k$ all over VISp, we find the single best $K$-plet $k$ 
whose $q(\mathbf{x}|k)$ is the most similar to $Q_{\rm GFP}(\mathbf{x})$.

Such a measure of similarity between two probability distributions $p$ and $q$ is given by the cross-entropy ${\cal H}(p,q)$ which is a minimum only if $p=q$. For our case, we want to find $k$ such that 
\beq
{\cal H}(k) = - \sum_{\mathbf{x}\in\text{VISp}} Q_{\rm GFP}(\mathbf{x}) \log\left[ q(\mathbf{x}|k)\right],
\eeq{cross}
is minimized among all $k\in {\cal K}$. Let $k=k^*$ be the optimal $K$-plet. Then, given the assumption/model that the gene tool has specificity of $K$ cell types, the expected fractions $f_s$ of subclasses $s\in k^*$ present all over VISp is
\beq
f_s =  \sum_{\mathbf{x}\in\text{VISp}} Q_{\rm GFP}(\mathbf{x}) \frac{P(s|\mathbf{x})}{\sum_{s'\in k^*} P(s'|\mathbf{x}) },\quad\text{for}\quad s\in k^*.
\eeq{frac}
For $s\notin k^*$, $f_s=0$.
The assumptions of specificity $K$ and the optimal $k^*$ from ${\cal K}$ are implicit, and it is to be understood that 
$f_s$ is the above equation is actually $f_s\left(k^*(K)\right)$.

\subsection{Model averaging}
The above step concludes the algorithm. But, one could go further to alleviate the model dependence on the choice of gene tool specificity $K$ by a weighted average of $f_s(k^*)$ for different $K$. Studies are needed on how best to perform the weighted average, but for this challenge, we do a simple model averaging as 
\beq
f_s = \frac{1}{K_{\max}} \sum_{K=1}^{K_{\rm max}} f_s\left(k^*(K)\right),
\eeq{modav}
for $K_{\max}=4$.

\end{document}

