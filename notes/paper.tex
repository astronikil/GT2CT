\documentclass[prd, onecolumn, superscriptaddress, nofootinbib, notitlepage, floatfix]{revtex4-1}
\usepackage{latexsym,amsmath,amssymb,amstext}
\usepackage{slashed}
\usepackage{epsfig}
\usepackage{float,xcolor}
\usepackage{graphicx}
\usepackage{dcolumn}

\newcommand{\tr}{{\rm tr\ }}

% Environment elements simplified
\newcommand{\be}{\begin{equation}}
\newcommand{\ee}{\end{equation}}
\newcommand{\bea}{\begin{eqnarray}}
\newcommand{\eea}{\end{eqnarray}}
\newcommand\bef{\begin{figure*}}
\newcommand\eef[1]{\label{fg:#1}\end{figure*}}
\newcommand\beq{\begin{equation}}
\newcommand\eeq[1]{\label{#1}\end{equation}}
\newcommand\beqa{\begin{eqnarray}}
\newcommand\eeqa[1]{\label{#1}\end{eqnarray}}
\newcommand\bet{\begin{table}}
\newcommand\eet[1]{\label{tb:#1}\end{table}}

\newcommand\fgn[1]{Figure \ref{fg:#1}}
\newcommand\eqn[1]{Eq.\ (\ref{#1})}
\newcommand\scn[1]{Section \ref{sec:#1}}
\newcommand\apx[1]{Appendix \ref{sec:#1}}
\newcommand\tbn[1]{Table \ref{tb:#1}}
\newcommand\nk[1]{\textcolor{red}{#1}}

% Recurring text elements
\newcommand\ie{{\sl i.e.\/}}
\newcommand\etc{{\sl etc.\/}}

% Rewriting rules
\newcommand{\mn}{{\rm min}}
\newcommand{\mx}{{\rm max}}
\newcommand{\st}{{\rm st}}
\newcommand{\lat}{\mathrm{lat}}
\newcommand{\phys}{\mathrm{phys}}

\begin{document}

\widetext

\title{Numerical hint for the $4\pi$-flux monopoles being irrelevant in $N=4$ QED$_3$}

\author{Nikhil\ \surname{Karthik}}
\email{nkarthik.work@gmail.com}
\affiliation{American Physical Society, Hauppauge, New York 11788}
\affiliation{Department of Physics, Florida International University, Miami, FL 33199}
\author{Rajamani\ \surname{Narayanan}}
\email{rajamani.narayanan@fiu.edu}
\affiliation{Department of Physics, Florida International University, Miami, FL 33199}



\begin{abstract}

We study the issue of relevance of monopole operators under
renormalization group flow in three-dimensional parity-invariant
noncompact QED with 4 flavors of massless two-component Dirac
fermions.  By means of a finite-size scaling analysis of the free
energy to introduce monopole-antimonopole pairs in  $N=4$ and $N=12$
flavor noncompact QED$_3$ computed using lattice simulation, we
estimate the infrared scaling-dimensions of monopole operators that
introduce $2\pi$ and $4\pi$ fluxes around them.  We first show that
the estimates for the monopole scaling dimensions to be consistent
with the large-$N$ expections for $N=12$ QED$_3$. Applying the same
procedure in $N=4$ QED$_3$, we estimate the scaling dimension of
$4\pi$ flux monopole operator to be $3.7(3)$, which allows the
possibility of the operator being irrelevant.  This finding offers
support to the scenario in which higher-flux monopoles are irrelevant
perturbations to certain nonbipartite lattice models which could
host a stable U(1) Dirac spin liquid phase by forbidding $2\pi$-flux
monopoles.

\end{abstract}

\maketitle

\section{Introduction}\label{sec:intro}

The characterization of the quantum numbers of monopoles in the continuum and on 
various lattices, and their impact on the infrared behavior of three-dimensional
quantum electrodynamics coupled to massless Dirac fermions have
been topics of interest in recent times. The presence or absence
of monopoles are expected to change the long-distance behavior of
QED$_3$ radically.  QED$_3$ without monopole excitations -- noncompact
QED$_3$ -- has been intensely studied using various methods, for example, using lattice
regularization~\cite{}, conformal bootstrap~\cite{}, Dyson-Shwinger
approaches~\cite{} \textcolor{red}{(Citations to be added to make 
people happy.)}.  Whereas consensus is yet to be reached on the
infrared fate of noncompact QED$_3$ coupled to few flavors of
massless fermions from independent nonperturbative methods, recent
lattice studies by the authors of this paper using exactly massless fermions
show strong numerical evidence for the infrared scale-invariance
of noncompact QED$_3$ with non-zero even number of massless flavors
(from finite-size scaling of low-lying Dirac
eigenvalues~\cite{Karthik:2015sgq,Karthik:2016ppr} and closer
resemblence of their eigenvalue distributions to those from a simple
conformal model~\cite{Karthik:2020shl}, and presence of power-law
correlators~\cite{Karthik:2016ppr,Karthik:2017hol}) On the other
hand, QED$_3$ with any number of monopoles -- compact QED$_3$ -- without
massless fermion content is well known to be
confined~\cite{Polyakov:1975rs,Polyakov:1976fu}. The nature of
compact QED$_3$ coupled to exactly massless fermions is yet unknown,
but one expects that there will be a non-zero critical number of
flavor above which the compact QED$_3$ will flow to an infrared
fixed point.

A convenient indirect method to probe the infrared conformality in
compact QED$_3$ is through monopole operator insertions in the corresponding
noncompact QED$_3$.  Even though monopoles do not arise dynamically
in the noncompact theory after the ultraviolet regulator is removed, 
one could subject the noncompact theory to monopolelike singular
boundary conditions at various space-time points;
for a flux $Q$ monopole, the total flux on surfaces enclosing the
point is $2\pi Q$ for integers $Q$. For fermions coupled to the
U(1) gauge fields, the extended Dirac string singularity is invisible, and the
insertion of the monopole behaves like the insertion of a composite
operator at the point. Hence, one defines the monopole operator
${\cal M}_Q(x)$ through its action of introducing  $2\pi Q$ flux
around the point $x$~\cite{Borokhov:2002ib}. At critical points of a U(1) lattice theory, 
one can find the scaling dimension of such monopole operators via the two-point functions,
\beq
\left\langle {\cal M}^\dagger_Q(x) {\cal M}_Q(0)\right\rangle \propto |x|^{-2\Delta_Q}.
\eeq{mono2pt}
The exponent $\Delta_Q$ is the scaling dimension of ${\cal M}_Q$.
Since criticality is approached only in the long-distance or the infrared limit of
QED$_3$, the above power-law scaling will be seen when the monopole
and antimonopole are separated by large distances.  If the infrared
dimension $\Delta_Q >3$, then flux-$Q$ monopoles are irrelevant
to the infrared end of the renormalization group flow.
%formally, an addition of a term proporional to ${ \cal M}_Q$ to the
%QED$_3$ action that creates any number of $Q$ monopoles should not
%change the infrared physics.  

A motivation for computing the scaling dimensions of monopoles in
the noncompact theory is
the interest in the infrared fate of compact QED$_3$ that is otherwise quite 
difficult to study on the lattice due to near-zero Dirac operator eigenmodes. 
In compact QED$_3$, monopoles of all
flux-$Q$ dynamically appear.  As $\Delta_Q$ is typically a monotonically
increasing function of $Q$, if $\Delta_1$ of the $Q=1$ monopole is
greater than 3 in an $N$ flavor noncompact QED$_3$, then one expects
a similar conformal behavior in the $N$-flavor compact QED$_3$ and
noncompact QED$_3$. In this way, the $Q=1$ monopole is expected to
be relevant only for $N<12$ based on large-$N$ and $4-\epsilon$ approximations~\cite{Pufu:2013vpa,Chester:2015wao}, and
further confirmed by lattice simulations in previous as well as in
the present work. By the above argument, the compact QED$_3$ is
expected to have a critical number of flavor $N\approx 12$.

Given the dominant role of the monopole creating the smallest $2\pi$
flux, a study of higher flux creating monopoles might not seem
significant.  However, the specific motivation for studying monopoles
that create $4\pi$ flux in this work is the following.  Recently,
there has been interest in a realization of compact QED$_3$ where
the dominant $Q=1$ monopole is disallowed due to ultraviolet
symmetries specific to certain lattices~\cite{Song:2018ial}.  Such
a version of $N=4$ compact QED$_3$ that is devoid of $Q=1$ monopole
is expected to be an effective field theory description of the
Heisenberg spin model in Kagom\'e and triangular lattices that is
expected to host a U(1) Dirac spin liquid
phase~\cite{Song:2018ial,Song:2018ccm}. The possibility of the
long-distance correlation in the DSL phase is tied to the infrared
conformality of the $N=4$ compact QED$_3$ in the absence of $Q=1$
monopoles; in other terms, to the infrared irrelevance of the
next-allowed $Q=2$ monopole operators in $N=4$ noncompact QED$_3$.
The nonperturbative determination of the scaling dimension of the
$Q=2$ monopoles in $N=4$ QED$_3$ using direct lattice simulation
is therefore the main aim of this paper

\section{Method}\label{sec:method}

\subsection{Lattice regulated noncompact QED$_3$ and monopole insertions therein}

The parity-invariant noncompact theory consists of Abelian gauge
fields $A_\mu(x)$ coupled to an even number of flavors, $N$, of
massless two-component Dirac fermions.  The gauge coupling $g^2$
in the theory has a mass dimension of 1, which makes the theory
super-renormalizable and we can use appropriate factors of $g^2$ to make all
masses and lengths dimensionless. We study the lattice regulated
version of the theory on a periodic box of dimensionless physical
volume $\ell^3$ that is discretized using a lattice of size $L^3$.
The continuum limit of the finite volume theory can be obtained by
extrapolating to $L\to\infty$ limit in different fixed physical
extents $\ell$.

We take a brief detour to formally define noncompact gauge theory
that is explicitly a U(1) gauge theory, and define monopole insertions
within this U(1) lattice gauge theory.  The Villain
formulation~\cite{Villain:1974ir} of the noncompact QED$_3$ can be
defined via the path integral,
\beq
Z= \int_{-\infty}^\infty \left(\prod_{x,\mu} d\theta_\mu(x)\right) \det{}^{N/2}\left[\slashed{C}\slashed{C}^\dagger\right] {\cal W}_g(\theta),
\eeq{latticepath}
where $\slashed{C}$ is a two-component lattice Dirac operator that
is coupled to the lattice gauge fields $\theta_\mu(x) = A_\mu(x)
\ell/L$ via the compact variable $U_\mu(x) = e^{i\theta_\mu(x)}$.
The theory is regulated in a parity-invariant manner by coupling
$N/2$ flavors are coupled to $\slashed{C}$, and the other $N/2$ to
$\slashed{C}^\dagger$. The contribution from the gauge sector is
${\cal W}_g(\theta)$ given by,
\beq
{\cal W}_g(\theta) = \sum_{\{N_{\mu\nu}\}}\exp\left[-\frac{L}{\ell}\sum_{x}\sum_{\mu>\nu}\left(F_{\mu\nu}(x) - 2\pi N_{\mu\nu}(x)\right)^2\right]\quad\text{where}\quad
F_{\mu\nu}(x) = \Delta_\mu\theta_\nu(x)-\Delta_\nu\theta_\mu(x).
\eeq{wgdef}
with the sum over configurations of integer values $N_{\mu\nu}(x)$
associated with the $(\mu,\nu)$-plaquette at site $x$.  Since both
${\cal W}_g(\theta)$ and the Dirac operator $\slashed{C}$ are
invariant under shifts $\theta_\mu(x) \to \theta_\mu(x) + 2\pi
n_\mu(x)$ for integers $n_\mu(x)$, the path-integral \eqn{latticepath}
is that of U(1) gauge theory coupled to fermions.  The magnetic
charge $Q(x)$ of the monopole at a site $x$ is
defined~\cite{DeGrand:1980eq} via the divergence of an integer-valued
current dual to $N_{\mu\nu}(x)$; that is,
\beq
Q(x) \equiv \frac{1}{2}\sum_{\mu,\nu,\rho}\epsilon_{\mu\nu\rho}\Delta_{\mu}N_{\nu\rho}(x).
\eeq{monocharge}
For the sake of brevity, we simply define a monopole at a point $x$
with a value of net flux as $2\pi Q$ as a flux-$Q$ monopole.
Depending on the constraints on the allowed values of $Q(x)$ in the
path-integral, which thereby corresponds to constraints on the
allowed configurations $\{N_{\mu\nu}\}$ in \eqn{wgdef}, one can
define different versions of QED$_3$.  The noncompact QED$_3$ is
the U(1) gauge theory with the constraint, $Q(x)=0$, at all $x$ in
the continuum limit of the lattice regulated theory. In this case,
by making use of the invariance of theory under $\theta_\mu \to
\theta_\mu + 2\pi n_\mu$ shifts, one can write down the path-integral
in the usual form without any sum over $N_{\mu\nu}$ as
\beq
Z_0 = \int_{-\infty}^\infty \left(\prod_{x,\mu} d\theta_\mu(x)\right) \det{}^{N/2}\left[\slashed{C}\slashed{C}^\dagger\right] e^{-\frac{L}{\ell} \sum_{x}\sum_{\mu>\nu}F_{\mu\nu}(x)^2}.
\eeq{normalpath}
We can define the path-integral $Z_Q$ with a flux-$Q$ monopole inserted at a point $x'$ and an antimonopole at a point $x'+r$ by subjecting to the constraint $N_{\mu\nu}(x)=N^{Q\bar Q}_{\mu\nu}(x;r)$ with
\beq
\frac{1}{2}\sum_{\mu,\nu,\rho}\epsilon_{\mu\nu\rho}\Delta_{\mu} N^{Q\bar Q}_{\nu\rho}(x;r) = Q \delta_{x,x'} - Q \delta_{x,x'+r}.
\eeq{Nconstraint}
The two-point function of a monopole at $x'$ and $x'+r$ is simply
the ratio, $Z_Q/Z_0$.

The universal aspects like the anomalous dimensions at the infrared
fixed point should not be sensitive to the exact details of the lattice operator
so long as the operator quantum numbers are captured correctly. Therefore, we
can choose the type of background flux to better capture the effect
of monopole operators.  Instead of introducing integer valued flux
$N^{Q\bar Q}_{\mu\nu}$ in the path-integral, we follow the approach
of Refs~\cite{Murthy:1989ps,Pufu:2013eda} to introduce a classical background gauge field
${\cal A}^{Q\bar Q}_{\mu}(x;r)$ that minimizes the pure gauge action
\beq
S_g = \left[ B_{\mu\nu}^{Q\bar Q}(x;r) - 2\pi N^{Q\bar Q}_{\mu\nu}(x;r)\right]^2; \qquad B_{\mu\nu}^{Q\bar Q}(x;r) = \Delta_\mu {\cal A}^{Q\bar Q}_{\nu}(x;r) - \Delta_\nu {\cal A}^{Q\bar Q}_{\mu}(x;r),
\eeq{bmunu}
on a periodic box.
We then define the path-integral in the presence of monopole insertions via,
\beq
Z_Q = \int_{-\infty}^\infty \left(\prod_{x,\mu} d\theta_\mu(x)\right) \det{}^{N/2}\left[\slashed{C}_W\slashed{C}_W^\dagger\right] e^{-\frac{L}{\ell} \sum_{x}\sum_{\mu>\nu}\left(F_{\mu\nu}(x)- Q B^{1\bar 1}_{\mu\nu}(x;r)\right)^2},
\eeq{normalpathQ}
using the fact that $B_{\mu\nu}^{Q\bar Q} = Q B_{\mu\nu}^{1\bar
1}$.  The above procedure has the advantage that the effect of
monopole can be completely removed from the path-integral in the
pure-gauge theory ($N=0$) by redefining the dynamical gauge field
$\theta_\mu(x) \to \theta_\mu(x) - {\cal A}^{Q\bar Q}_\mu(x;r)$.
Therefore any non-zero effect of the monopole at finite non-zero
$N$ can arise only due to the presence of massless fermions.  This
procedure has been put to test previously in the free fermion
theory~\cite{Karthik:2018rcg}, critical point of the 3d
XY-model~\cite{Karthik:2018rcg}, and large-$N$
QED$_3$~\cite{Karthik:2019mrr}.

\subsection{Monopole correlator and finite-size scaling}

The lattice monopole two-point function is given by 
\beq
G_\lat(r,\ell,L) = \frac{Z_Q}{Z_0}.
\eeq{baretwopoint}
As with correlators of regular composite operators composed of local
fields, we assume that the lattice correlator $G_\lat$, which is
in units of lattice spacing $a=\ell/L$, can be converted into a
correlator in physical units $G_\phys$ by a multiplicative factor;
namely,
\beq
G_\phys(r,\ell,L) = a^{-2D_Q^{(N)}} G_\lat(r,\ell,L),
\eeq{renormalization}
where $D_Q^{(N)}$ is the ultraviolet exponent governing the monopole
correlator at short distances.  We will discuss more on the conversion
from the lattice to the physical correlator when presenting the
results from our numerical calculation.  The physical continuum
correlator, $G_\phys(r,\ell)$, after extrapolating to $L\to\infty$ will
show scale-invariant behavior at large separations $|r|$ and $\ell$
as
\beq
G_\phys(r,\ell) = \frac{1}{|r|^{2\Delta^{(N)}_Q}} {\cal G}\left(\frac{|r|}{\ell}\right)\quad\text{as}\quad |r|\to\infty.
\eeq{scaleinvcorr}
The exponent $\Delta^{(N)}_Q$ that governs the long-distance
correlator is the infrared scaling-dimension of the monopole operator
that we are seeking.  By keeping the ratio $|r|/\ell = \rho$ fixed
as $\ell$ is increased,
\beq
G_\phys(|r|=\rho\ell,\ell) \propto \frac{1}{\ell^{2\Delta^{(N)}_Q}}\quad\text{as}\quad \ell\to\infty.
\eeq{scaleinvcorr2}
We will follow this procedure in this work, and keep the monopole-antimonopole
separation proportional to box size, thereby reducing the
determination of the infrared scaling dimension to a finite-size
scaling analysis.

\bef
\centering
\includegraphics[scale=0.6]{plot_w_vs_zeta_nf2_L24.pdf}
\includegraphics[scale=0.6]{plot_w_vs_zeta_nf6_L24.pdf}
\caption{Representative data points and interpolation for the lattice
determined derivative of free energy, $W(\zeta)$, with respect to
the auxilliary parameter $\zeta$. The top and bottom panels show
results for  $W(\zeta)$ at different $\ell$ on $L=24$ lattice in
$N=4$ and 12 noncompact QED$_3$ respectively. The red data points
are the actual Monte Carlo determinations. The black bands are the
spline interpolations to the data. }
\eef{wversusz}

\subsection{Implementation of the numerical calculation}

We studied noncompact theory with $N=4$ and $N=12$ flavors on
periodic Euclidean boxes at multiple values of physical extents
$\ell=$ 4, 8, 16, 24, 32, 48, 64, 96, 128, 144, 160 and 200.  We
discretized them on lattices of volume $L^3$ with $L=12, 16,20,24$
and 28. We used Wilson-Dirac operator $\slashed{C}_W$ that is coupled
to 1-step HYP-smeared gauge field. We tuned to the massless point
by tuning the bare Wilson fermion mass $m_w$ so that the first
eigenvalue $\Lambda_1^2(m_w)$ of $\slashed{C}_w\slashed{C}_w^\dagger$
is minimized as a function of $m_w$. More details on the two-component
Wilson-Dirac operator and its mass tuning can be in our earlier
work in Ref~\cite{Karthik:2015sgq}.

We chose the displacement vector $r$ between the flux-$Q$ monopole
and antimonopole to be along one of the axis; namely, the
three-vector $r=(0,0,t )$ for $t=aT$ and integers $T$.  We kept
$\rho = t/\ell=1/4$, an arbitrary choice in the work to simplify
the analysis to a finite-size scaling one as explained before.  For
this choice of on-axis $r$, a natural choice for $N_{\mu\nu}^{1\bar
1}$ that satisfies \eqn{Nconstraint} is $N_{12}^{1\bar 1}(0,0,x_3)
= 1$ for $x_3\in[1,T]$, and all other $N_{\mu\nu}^{1\bar 1}$ are
set to 0.  Such a choice can be changed arbitrarily by shifts
$N_{\mu\nu}^{1\bar 1} \to N_{\mu\nu}^{1\bar 1} + \Delta_\mu n_\nu
- \Delta_\nu n_\mu$ for integers $n_\mu(x)$, that moves and bends
the Dirac string (the column of plaquettes with $2\pi$ flux)
keeping the location of monopole and antimonopole fixed; however
such variations are unimportant in the U(1) theory, and therefore,
the simplest choice above for $N_{12}^{Q\bar Q}$ suffices. With this
choice of $N_{\mu\nu}^{1\bar 1}$, we determined the background field
${\cal A}^{1\bar 1}_\mu(x)$, and the field tensor $B^{1\bar
1}_{\mu\nu}(x)$ in the periodic $L^3$ box by numerically minimizing
the action \eqn{bmunu}. From this, the background field for any value
of $Q$ can be obtained as $Q B^{1\bar 1}_{\mu\nu}(x)$.

The effect of $B^{Q\bar Q}_{\mu\nu}$ is exponentially suppressed
in \eqn{normalpathQ}, and therefore, it is hard to compute $Z_Q$
as an expectation value in $Q=0$ theory.  Instead, we follow the
approach in Ref~\cite{Karthik:2018rcg}, and computed the logarithm
of the above correlator, which is nothing but the free energy in
lattice units to introduce monopole-antimonopole pair, as
\beq
F_\lat(r,\ell,L)\equiv -\log(G_\lat(r,\ell,L)) = \int_0^Q W(\zeta) d\zeta,
\eeq{barefreeenergy}
where 
\beq
W(\zeta) = -\frac{1}{Z_\zeta}\frac{\partial}{\partial\zeta}Z_{\zeta},
\eeq{wdef}
and $Z_\zeta$ is the extension of the path-integral in \eqn{normalpathQ}
by the replacement $Q\to \zeta$ for real values of $\zeta$.  We
have simply differentiated $F_\lat$ with respect to an auxilliary
variable $\zeta$ and integrated it back again. The reason behind
doing so is that the quantity $W(\zeta)$ is computable as expectation
values, $\langle\cdots\rangle_\zeta$, in the Monte Carlo simulation
of the $Z_\zeta$ path-integral; namely,
\beq
W(\zeta) = \frac{2L}{\ell}\sum_{\mu>\nu}\sum_{x} B_{\mu\nu}^{1\bar 1}(x;r) \left\langle F_{\mu\nu}(x) - \zeta B_{\mu\nu}^{1\bar 1}(x;r)\right\rangle_\zeta.
\eeq{wzetameas}
We used 40 different equally spaced values of $\zeta\in[0,2]$.  At
each value of $\zeta$, we performed independent hybrid Monte Carlo
(HMC) simulation of $Z_\zeta$ to compute $W(\zeta)$ numerically.
From each thermalized HMC run, we generated between 15K to 30K
measurements of $W(\zeta)$. By using Jack-knife analysis, we took
care of autocorrelations in the collected measurements.

\section{Results}\label{sec:results}


\subsection{Determination of free energy}


From the Monte Carlo simulation, we collected the data for $W(\zeta)$
from the relation in \eqn{wzetameas}.  In \fgn{wversusz}, we show
the numerically determined $W(\zeta)$ (the red circles in the panels)
as a function of $\zeta$ at all $\ell$ on a fixed $L=24$ lattice.
We show the data from $N=4$ and 12 flavor theories in the set of
top and bottom panels respectively. The actual simulation points
span $\zeta\in[0,2]$.  In order to perform the needed integration
in \eqn{wdef}, we interpolate the data between 0 and 2 using cubic
spline first. The black bands in the figures overlayed over the
data points are such interpolations.  By choosing the endpoint of
the integration of the interpolated data to be either 1 or 2, we
get the free energy to introduce the $Q=2$ monopole-antimonopole
pair, or the $Q=1$ monopole-antimonopole pair respectively.  Thus,
without any extra computational cost, we study both $Q=1$ and $Q=2$
monopole in this paper.


The numerical integrations of the data result in the lattice free
energy, $F_\lat(\ell;L)$. In \fgn{fbare}, we show the $\ell$
dependence of $F_\lat(\ell;L)$ for $Q=1,2$ and $N=4,12$. The different
colored symbols within the panels are the data from different $L$.
At first sight,  the apparent decrease in $F_\lat(\ell;L)$ with
increase in $\ell$ at various fixed $L$ might strike one to be
against expectation.  The reason behind such a behavior of the
lattice free energy is because the lattice spacing $\ell/L$ at
various $\ell$ at a fixed $L$ also changes when $\ell$ is increased.
The conversion of the lattice free energy to physical units should
restore a physically meaningful increasing tendency of the free
energy with the monopole-antimonopole separation, and also be able
to bring an approximate data collapse of the free energy from
different $L$.

\bef
\centering
\includegraphics[scale=0.55]{plot_fbare_nf2_Q1.pdf}
\includegraphics[scale=0.55]{plot_fbare_nf6_Q1.pdf}
\includegraphics[scale=0.55]{plot_fbare_nf2_Q2.pdf}
\includegraphics[scale=0.55]{plot_fbare_nf6_Q2.pdf}

\caption{The free energy, $F_\lat(\ell,L)$, in lattice units is
shown as a function of physical box size, $\ell$.  The values of
$(Q,N)$ in the panels from left to right correspond to $(1,4),
(1,12), (2,4), (2,12)$ respectively. In each panel, the data at
$L=12,16,20,24$ and 28 are shown.
}
\eef{fbare}


We converted lattice correlator $G_\lat$ to physical $G_\phys$ by
a lattice spacing dependent factor $a^{-2D_Q^{(N)}}$ as explained
in \eqn{renormalization}.  Equivalently, the conversion between the
lattice and physical free energies is brought about by an additive
$2D_Q^{(N)} \log(a)$ term.  For regular composite operators built out 
of the field operators $\psi$ and $A_\mu$ such
as a fermion bilinear $\bar\psi_i\bar\psi$, the ultraviolet dimensions
follow from the power-counting arguments; taking the example of
fermion bilinear, they are of ultraviolet dimension of two, and the
lattice bilinear can be converted to physical units by a factor
$a^{-2}$. However, a monopole operator at $x$ is not expressible in such a simple 
form in terms of the fermion and gauge fields at $x$, and a
power-counting cannot be performed. Therefore, we have to rely on empirical
determination of the UV exponent $2D_Q^{(N)}$. As the exponent $D_Q^{(N)}$
should govern the short-distance behavior of the monopole-antimonopole
correlator, we estimated $D_Q^{(N)}$ from a leading logarithmic behavior,
\beq
F_\lat(L) = F_0 + 2 D_Q^{(N)} \log(L), 
\eeq{uvdimscaling}
of the lattice free energy at a fixed small lattice spacing $a=1/7$
corresponding to small box-sizes $a L \le 4$ on $L=12$ to 28. This
is equivalent to short monopole-antimonopole separations $t = aL/4
\le 1$ on such boxes where the above $\log(L)$ dependence could arise.  
In \fgn{uvdim}, we show such a $\log(L)$
dependence of $F_\lat(L)$ at $a=1/7$ for $Q=1,2$ and $N=4,12$. The
red data points are from the Monte Carlo simulations on $L=$ 12,
16, 20, 24 and 28 lattices.  For $L\in[16,28]$, the data is consistent
with a $\log(L)$ dependence of the free energy.  The $L=12$ lattice
point is slightly off from the logarithmic behavior, which suggests
the presence of lattice artifacts at such close $t=3a$ separation
between the monopole and the antimonopole. The black band is the
best fit of \eqn{uvdimscaling} to the data using $F_0$ and $D_Q^{(N)}$
as fit parameters. Our best empirical estimates of the ultraviolet
dimensions are $D_1^{(4)}=1.85(11)$, $D_1^{(12)}=5.50(25)$,
$D_2^{(4)}=4.67(16)$ and $D_2^{(12)}=14.01(36)$ respectively.

\bef
\centering
\includegraphics[scale=0.55]{plot_uvdim_nf2_Q1.pdf}
\includegraphics[scale=0.55]{plot_uvdim_nf6_Q1.pdf}
\includegraphics[scale=0.55]{plot_uvdim_nf2_Q2.pdf}
\includegraphics[scale=0.55]{plot_uvdim_nf6_Q2.pdf}

\caption{ Estimation of UV scaling dimension $D^{(N)}_Q$ of the
flux-$Q$ monopoles from the scaling of the lattice free energy $F_\lat(\ell,L)$
in $N$-flavor theory with box size $L$ at fixed small lattice spacing
$\ell/L=1/7$.  The panels show $F_\lat(\ell,L)$ as a function of $L$.
The values of $(Q,N)$ in the panels from left to right correspond
to $(1,4), (1,12), (2,4), (2,12)$ respectively.  The data points
are the lattice determined values of $F_\lat$.  The black bands are
the $\log(L)$ fits to the data. }
\eef{uvdim}

\bef
\centering
\includegraphics[scale=0.55]{plot_fren_nf2_Q1.pdf}
\includegraphics[scale=0.55]{plot_fren_nf6_Q1.pdf}
\includegraphics[scale=0.55]{plot_fren_nf2_Q2.pdf}
\includegraphics[scale=0.55]{plot_fren_nf6_Q2.pdf}

\caption{The free energy $F_\phys(\ell,L)$ converted to physical units
is shown as a function of
$\ell$. The data points at fixed $L$ are shown using the different
colored symbols. The top-left and top-right panels show $Q=1,N=4$
and $Q=1,N=12$, whereas the bottom-left and bottom-right panels
show $Q=2,N=4$ and $Q=2,N=12$ data sets respectively.}
\eef{fren}

\bef
\centering
\includegraphics[scale=0.55]{plot_cont_nf6_Q1.pdf}
\includegraphics[scale=0.55]{plot_cont_nf6_Q2.pdf}
\includegraphics[scale=0.55]{plot_cont_nf2_Q1.pdf}
\includegraphics[scale=0.55]{plot_cont_nf2_Q2.pdf}

\caption{The continuum estimates of the $F_\phys(\ell)$ through
$1/L$ extrapolations of $F_\phys(\ell,L)$ at different fixed physical
box sizes $\ell$ noted beside the data points. The $1/L$ fits were
made over the range $L\in[16,28]$. The bands show the extrapolations
resulting from the fits. The values of $(Q,N)$ for the panels from 
left to right correspond to
$(1,12)$, $(12,2)$, $(4,1)$ and $(4,2)$ respectively.}
\eef{cont}

Using the determined $F_\lat(\ell;L)$ and the best fit values of
$D_Q^{(N)}$ in the previous analysis, we obtained $F_\phys(\ell;L)$
as
\beq
F_\phys(\ell;L) = F_\lat(\ell;L) + 2D_Q^{(N)} \log\left(\frac{\ell}{L}\right),
\eeq{freelattoren}
We propagated the statistical errors in $F_\lat(\ell;L)$ and the
estimated $D_Q^{(N)}$ into the determination of $F_\phys$ by adding
the errors in quadratures.  In \fgn{fren}, we show $F_\phys(\ell;L)$
as a function of $\log(\ell)$ for $Q=1,2$ in $N=4$ and 12 flavor
theories. In each panel, we show the results of $F_\phys(\ell;L)$
from $L=12$, 16, 20, 24 and 28 together. First, we notice that
$F_\phys(\ell)$ increases monotonically as we expected.  Second,
the lattice-to-physical units `renormalization factor', $a^{-2D_Q^{(N)}}$
has caused a near data collapse of the $F_\phys(\ell;L)$ from
multiple $L$. The residual $L$ dependencies at fixed $\ell$ needs
to be removed by extrapolating to $L\to\infty$ as we discuss below.


In \fgn{cont}, we show the residual $1/L$ dependence of $F_\phys(\ell;L)$
for $Q=1$ and 2 monopoles in $N=4$ and 12 theories.
The data points differentiated by
their colors have a fixed value of $\ell$, and they have to be
extrapolted to $L\to\infty$ to estimate the continuum limit
$F_\phys(\ell)$ in that physical box size. We perform the extrapolation
using a simple Ansatz, $F_\phys(\ell;L) = F_\phys(\ell) + k(\ell)/L$
with $F_\phys(\ell)$ and $k(\ell)$ fit parameters, to describe the
$L$-dependence of $F_\phys(\ell;L)$ for $L\in[16,28]$.  Such a fit
was capable of describing the $L$-dependence well with $\chi^2/{\rm
dof}<1$ in most cases. For $\ell=144,160,200$ in $N=12$ theory, by
mistake, we did not produce the $L=28$ lattice data.  Therefore, we
performed the extrapolation only using $L=16,20$ and 24 data sets
in such cases resulting in a comparatively larger statistical error
in their extrapolated values.  The various colored bands in \fgn{cont}
show such $1/L$ extrapolations at various fixed $\ell$.  We will
use the extrapolated $F_\phys(\ell)$ in the discussion of infrared
dimensions of monopole operators in the next subsection.

\subsection{Estimation of infrared scaling dimensions}


First, we discuss the scaling dimensions in $N=12$ theory.  Due to
the relatively large value of $N$, this serves as a test case to
see if the values obtained for the scaling dimension agree approximately
with the large-$N$ expectations.  In \fgn{nf6q12cfren}, we show the
dependence of $F_\phys(\ell)$ as a function of $\log(\ell)$ for the
$N=12$ case.  In the left and right panels, we show the $Q=1$ and
$Q=2$ monopole free energies respectively.  The black points in the
panels are our estimates for the continuum limits of $F_\phys(\ell)$,
as obtained in \fgn{cont}. For comparison, we also show the data
points for $F_\phys(\ell)$ from $L=24$ lattice before performing
any continuum extrapolation.  One expects a simple $\log(\ell)$
dependence only in the large-box limit, corresponding to large
separation between the monopole and the antimonopole. Within the
statistical errors, we see such a $\log(\ell)$ dependence for
$\ell\ge 32$. We fitted
\beq
F_\phys(\ell) = f_0 + 2\Delta_Q^{(N)}\log(\ell),
\eeq{asymfitform}
using a constant $f_0$, and the infrared scaling dimension
$\Delta_Q^{(N)}$ as fit parameters over two ranges $\ell\in[32,200]$
and $\ell\in[48,200]$ to check for systematic dependence on fit
range.  The underlying lattice data for $F_\lat$ are statistically
independent at different $\ell$, but \eqn{freelattoren} introduces
correlations between different $\ell$ due to the commonality of the
second term in \eqn{freelattoren}. We found the covariance matrix
of the data for $F_\phys$ at different $\ell$ close to being singular
making the minimization of correlated $\chi^2$ to be not practical,
and we resorted to uncorrelated $\chi^2$ fits; this is an approximation
made in this study. We determined the errors on fit parameters using
Jack-knife method.  For the $N=12$ theory under consideration, we
determined $\Delta_1^{(12)}$ and $\Delta_2^{(12)}$ in this way. We
show the resultant fits over $\ell\in[32,200]$ and $[48,200]$ as
the blue and magenta error bands respectively in the two panels of
\fgn{nf6q12cfren}.  The slope of the $\log(\ell)$ behavior from the
fit over $\ell\in[48,200]$ give
\beq
\Delta_1^{(12)}=2.81(66)\quad\text{and}\quad \Delta_2^{(12)}=8.2(1.0),
\eeq{n12q12deltavals}
for $Q=1$ and $Q=2$ monopoles respectively. The $\chi^2/{\rm dof}$
for the two fits are 1.1/6 and 3.2/6 respectively, which are smaller
than the typical value of around 1 due to the uncorrelated nature
of the fit.  By using a wider range of $\ell$ starting from a smaller
$\ell=32$, we found $\Delta_1^{(12)}=2.91(41)$ and
$\Delta_2^{(12)}=8.91(54)$ showing only a mild dependence on the
fit range.  The large-$N$ expectations~\cite{Pufu:2013vpa,Dyer:2013fja}
for these two scaling dimensions are $\Delta_1^{(12)}=3.1417$ and
$\Delta_2^{(12)}=7.882$.  We see that the estimates from fit over
$\ell\in[48,200]$ to be quite consistent with the large-$N$ expectation
well within 1-$\sigma$ error.  The more precise estimate of
$\Delta_2^{(12)}$ from the fit over $\ell\in[32,200]$ is slightly
higher than the large-$N$ value at the level of 2-$\sigma$, and is
more probable to be a systematic effect due to using smaller $\ell=32$
in the fit rather than arise due to genuine higher $1/N$ corrections.
It is reassuring that our numerical method obtains values for
$\Delta_Q^{12}$ consistent with the large-$N$ expectations, which
will add credence to the results from the method at smaller $N$ to
be discussed next.  As a minor note, by comparing to the slope of
the red points, we see that continuum extrapolation at all $\ell$
was essential, without which we would have overestimated the values
of $\Delta_Q^{(12)}$ by instead fitting the $\ell$ dependence at a
fixed $L$.

\bef
\centering
\includegraphics[scale=0.75]{plot_cfren_nf6_Q1.pdf}
\hskip 3em
\includegraphics[scale=0.75]{plot_cfren_nf6_Q2.pdf}

\caption{ The physical free energy $F_\phys(\ell)$ in $N=12$
QED$_3$ is shown as a function of $\log(\ell)$ for $Q=1$ monopole
in the left panel, and for $Q=2$ monopole in the right panel.  The
black points are the continuum estimates and the red points are
from a fixed $L=24$. The blue and magenta bands are the $\log(\ell)$ fit over
$\ell\in[32,200]$ and $[48,200]$ respectively.}

\eef{nf6q12cfren}

\bef
\centering
\includegraphics[scale=0.75]{plot_cfren_nf2_Q1.pdf}
\hskip 3em
\includegraphics[scale=0.75]{plot_cfren_nf2_Q1_sub3.pdf}

\caption{ (Left panel) The free energy $F_\phys$ is shown as a
function of $\log(\ell)$ for $Q=1$ monopole in $N=4$ theory.  The
black points are the continuum estimates and the red points are
from fixed $L=28$ lattice.  The blue and magenta bands are the
$\log(\ell)$ fit over $\ell\in[32,200]$ and $[48,200]$ respectively.
The slopes of the fits give the scaling dimension $\Delta_1^{(4)}$.
(Right panel) The difference from the expectation of the marginally
relevant scaling with $\Delta_c=3$ is explored. The data points and
the fitted bands in the left panel at larger $\ell$ are replotted
as the difference, $F_\phys(\ell)-2\Delta_c\log(\ell)$. The horizontal
dashed line is shown to compare the residual slope in the data with.
}

\eef{nf2q1cfren}

First, we consider the $Q=1$ monopole in the $N=4$ theory. We looked
at this case in our earlier work~\cite{Karthik:2019mrr}; the variation
in the present study is the usage of higher statistics, differences
in the sampled values of $\zeta$ to cover up to $\zeta=2$, and the
incorporation of dedicated continuum limits at each fixed $\ell$
instead of using a simpler one parameter characterization of $1/L$
effects at all $\ell$ used in the earlier work.  In the left panel
of \fgn{nf2q1cfren}, we show the $\log(\ell)$ dependence of the
free energy in the $N=4$ theory. The black points are the continuum
expectations, whereas the red ones are the data from the largest
$L=28$ lattice. Again, we see a simple $\log(\ell)$ behavior is
consistent with the data from boxes with $\ell\in[32,200]$. The fit
to the functional form \eqn{asymfitform} over a range $\ell\in[48,200]$
gives a slope of
\beq
\Delta_1^{(4)}=1.28(26),
\eeq{n4q1deltavals}
with a $\chi^2/{\rm dof}=1.2/6$.  We show the resulting fit as the
magenta error-band in \fgn{nf2q1cfren}. This is consistent with the
estimate $\Delta_1^{(4)}=1.25(9)$ from our earlier work. When we
include the smaller $\ell=32$ in the fit (shown as blue band), we
find $\Delta_1^{(4)}=1.27(13)$ pointing to a very mild dependence
on fit range. Clearly, $\Delta_1^{(4)}$ is smaller than the
marginal value $\Delta_c=3$, which makes the $Q=1$ monopole operator 
relevant along the renormalization group flows of $N=4$ QED$_3$. We can see
the relevance of $Q=1$ monopole operator without any fits by
plotting the difference $F_\phys(\ell)-2\Delta_c\log(\ell)$.  If the
operator is relevant, we should see a negative slope in the above
difference. In the right panel of \fgn{nf2q1cfren}, we show the
$\log(\ell)$ dependence of the difference,
$F_\phys(\ell)-2\Delta_c\log(\ell)$, with $\Delta_c=3$; it is a
simple replotting of the data and fits in the left panel.  We see
a clear negative slope in the data, and reach the same conclusion
about the relevance of $Q=1$ monopole in $N=4$ QED$_3$.  This bring
us to the main motivation for the present work; is the $Q=2$ monopole
operator also relevant in $N=4$ QED$_3$?


\bef
\centering
\includegraphics[scale=0.75]{plot_cfren_nf2_Q2.pdf}
\hskip 3em
\includegraphics[scale=0.75]{plot_cfren_nf2_Q2_sub3.pdf}

\caption{ (Left panel) The free energy $F_\phys$ is shown as a
function of $\log(\ell)$ for $Q=2$ monopole in $N=4$ theory.  The
black points are the continuum estimates and the red points are
from fixed $L=28$ lattice.  The blue and magenta bands are the
$\log(\ell)$ fit over $\ell\in[32,200]$ and $[48,200]$ respectively.
The slopes of the fits give the scaling dimension $\Delta_2^{(4)}$.
(Right panel) The difference from the expectation of the marginally
relevant scaling with $\Delta_c=3$ is explored. The data points and
the fitted bands in the left panel at larger $\ell$ are replotted
as the difference, $F_\phys(\ell)-2\Delta_c\log(\ell)$. The horizontal
dashed line is shown to compare the residual slope in the data with.
}
\eef{nf2q2cfren}

\bef
\centering
\includegraphics[scale=0.8]{plot_delta.pdf}

\caption{ The scaling dimension of $Q=2$ monopole is a plotted as
a function of $Q=1$ monopole using their estimates for $N=2,4$ and
12 QED$_3$.  The black crosses are the large-$N$ expectations.  The
filled red cicles are the estimates in this paper obtained from
finite-size scaling analysis of the data over $\ell\in[48,200]$.
The open red circles are obtained by fitting the data over
$\ell\in[32,200]$.  The red vertical band is using the estimate of
$\Delta^{(2)}_1$ from our earlier work.  The dashed lines indicate
the critical value of $\Delta_c=3$ for the two scaling dimensions.
}

\eef{delta}


In left panel of \fgn{nf2q2cfren}, we show the $\log(\ell)$ dependence
of $F_\phys(\ell)$ for $Q=2$ monopole in $N=4$ QED$_3$. As in the
previous cases we discussed, given the statistical errors, the
finite-size dependence of $F_\phys(\ell)$ for $\ell\ge 32$ is
consistent with a simple $\log(\ell)$ behavior. The magenta band
shows the fit using such a $\log(\ell)$ fit in \eqn{asymfitform}
to the data with $\ell\ge48$.  The value of the slope again gives
the scaling dimension. From the best fit values, we estimate that
scaling dimension of $Q=2$ monopole in $N=4$ QED$_3$ as
\beq
\Delta_2^{(4)}=3.73(34),
\eeq{n4q2deltavals}
with $\chi^2/{\rm dof}=1.4/6$.  Thus, $\Delta_2^{(4)}>3$  with a
weak statistical significance of about 2-$\sigma$.  If we start the
fit from a smaller $\ell=32$, we find a similar value
$\Delta_2^{(4)}=3.65(21)$ with a smaller error.  At $O(1/N)$ in the
large-$N$ expansion, $\Delta_2^{(4)} = 2.498$. Our data allows the
possibility that either by the importance of higher $1/N$ orders
in the large-$N$ expansion or by a breakdown of such an expansion
for $N=4$, the value of $\Delta_2^{(4)}$ could be larger than 3,
and make it irrelevant in the infrared.  As we explained in the
previous case of $Q=1$ monopole, the data is consistent with
$\Delta_2^{(4)}>3$ without fitting, we replot the data as a difference
$F_\phys(\ell) - 2\Delta_c\log(\ell)$, where the second term
corresponds to the expected slope at a marginal dimension $\Delta_c=3$.
In the right panel of \fgn{nf2q2cfren}, we show this difference
over a range of larger $\ell$.  In this plot, if the $Q=2$ monopole
was relevant, one should see a $\log(\ell)$ dependence with a
negative slope. The trend in the data points to a positive slope,
indicating the consistency of our data with $Q=2$ monopole being
irrelevant.  As a final remark, we note that the behavior of
$F_\phys(\ell)$ with $\Delta_2^{(4)}>3$ is not strongly dependent
on the continuum extrapolation that we performed.  The red points
in the two panels of \fgn{nf2q2cfren} are the free energies at
different $\ell$ on the largest $L=28$ lattice.  From the slope of
the red points, we see that we would have reached an even stronger
conclusion that $\Delta_2^{(4)}>3$ from that data alone. Therefore,
the effect of $L\to\infty$ extrapolation has been to make that
conclusion weaker.


We collect the results of $\Delta_Q^{(N)}$ in \fgn{delta}. We plot
$\Delta_2^{(N)}$ as a function of $\Delta_1^{(N)}$, making the
dependence on $N$ implicit, as it is convenient for usage in similar
exclusion plots obtained from numerical bootstrap (for
example, see Refs~\cite{Albayrak:2021xtd,He:2021sto}.)  The solid red
points in \fgn{delta} are the values determined in this paper using
fits over data from $\ell\in[48,200]$.  To show the systematic
artefacts in the estimate, we also show the estimates from fits
over data from $\ell\in[32,200]$ as the open circles.  In the
previous work~\cite{Karthik:2019mrr}, we determined only the value
of $\Delta_1^{(2)}$ in  $N=2$ QED$_3$.  Therefore, we show a red
band in \fgn{delta} to indicate lack of data for $\Delta_2^{(2)}$.
We show the large-$N$ expectation for $\Delta_2^{(N)}$ versus
$\Delta_1^{(N)}$ as the black crosses.  As we discussed before, the
top-right red point from $N=12$ QED$_3$ is consistent with the
large-$N$ expectation.  The vertical and horizontal dashed lines
in the figure indicate the marginal values of $\Delta_1=3$ and
$\Delta_2=3$ respectively.  The data point from $N=12$ lies at the edge
of $\Delta_1=3$, which indicates that $N=12$ QED$_3$ is close to being the
critical flavor below which $Q=1$ monopole becomes irrelevant. As
pointed out in~\cite{Pufu:2013vpa}, one could conjecture that the
critical flavor that separates the mass-gapped and conformal infrared
phases of $N$-flavor compact QED$_3$, where all flux-$Q$ monopoles
can freely arise, is around $N\approx 12$. The important finding
from this paper is that the $N=4$ data point in \fgn{delta} lies
above the critical horizontal line, albeit with a weaker statistical
significance of about 2-$\sigma$ (or 3-$\sigma$ if one bases the conclusion on 
the red open point).  We also see that the estimated
location of the $N=4$ data point in the plot is quite robust with
respect to change to the fitted range of $\ell$.  Thus, our data
cannot rule out the scenario where $Q=2$ monopole remains irrelevant
along the renormalization group flow even for the
$N=4$ theory.


\section{Conclusions}

The motivation for this study was the question of infrared relevance
of the $Q=2$ monopole operators in QED$_3$ coupled
to massless $N=4$ Dirac fermion flavors.  We used numerical lattice
simulations of noncompact QED$_3$ coupled to $N=4$ and $N=12$ flavors
of Wilson-Dirac fermions fine-tuned to the massless point. We
estimated the infrared scaling dimensions of $Q=1$ and $Q=2$ monopoles
in the $N=4$ and 12 theories from the finite-size scaling analysis
of free-energy required to introduce the $Q=1$ and 2 monopole-antimonopole
pairs in the two theories.  We validated the method in $N=12$ theory
first where the values of the $Q=1$ and 2 scaling dimensions would
be expected lie closer to the values obtained from first-order
large-$N$ expansion. Then, by applying to the $N=4$ theory, we found
our best estimate for $Q=2$ scaling dimension to $3.73(34)$, which
is consistent with being greater than the marginal value of $3$.
Thus, our result favors, and certainly cannot rule out, the possibility
of $Q=2$ monopole operators being irrelevant at the infrared fixed
point of $N=4$ QED$_3$.  We summarize our results for the scaling
dimensions in \fgn{delta} that shows the dimension of $Q=1$ monopole
as a function of the dimension of $Q=2$ monopole. It would be
interesting to compare this to similar determinations
from other methods.

As argued in Refs~\cite{Song:2018ial,Song:2018ccm}, the irrelevance
of $Q=2$ monopole operators could imply the possibility of hosting
a a stable U(1) Dirac spin liquid phase in non-bipartite lattices, such
as on the triangular and Kagom\'e lattice--on such lattices, it has
been argued~\cite{Song:2018ial,Song:2018ial} that the $Q=1$ monopoles
are disallowed, and the most important perturbation could be the
$Q=2$ monopoles.  The findings from our numerical study cannot rule
out this exciting possibility.

\bibliography{paper.bib}

\end{document}  
